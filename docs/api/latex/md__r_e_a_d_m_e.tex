\href{https://travis-ci.org/tpeulen/chinet}{\tt } \href{https://www.codacy.com?utm_source=github.com&amp;utm_medium=referral&amp;utm_content=tpeulen/chinet&amp;utm_campaign=Badge_Grade}{\tt } \href{https://conda.anaconda.org/tpeulen}{\tt } \href{https://anaconda.org/tpeulen/chinet}{\tt } \href{https://anaconda.org/tpeulen/chinet}{\tt }

\subsection*{General description}

chinet is a C++ library to create optimize, sample, and archive global models. A global model is a model, that unites multiple data-\/sets and seeks for a joint description of the united dataset.

Global models can unite datasets of the same kind or datasets of different types. A typical examples of a global model in fluorescence experiments is the joint description of multiple fluorescence correlation curves in a titration and the joint description of multiple fluorescence decay curves reporting on F\+R\+ET in a biomolecular structure by a single structural model.

Computing a gobal model with a large diverse set of different data can be computationally expensive. To reduce the computational costs and to decrease the evaluation time of a global model defined by chinet, the mutual dependencies of the model parameters are modeled by a graph structure that connects \char`\"{}computing nodes\char`\"{}. When a a set of parameters is changed only computing nodes that are affected by these changes are evaluated. Independent nodes are evaluated in parallel.

The state of the evaluation graph can be written to a database for documentation purposes and reconstructed using unique identifies provided by the database.

chinet is N\+OT intended as ready-\/to-\/use software for specific application purposes.

\subsection*{Goals}


\begin{DoxyItemize}
\item reactive dataflow model framework
\item fast inter computation node communication
\item define and store models jointly with associated data identifies in data base.
\item Low memory footprint (keep objective large datasets, e.\+g. F\+L\+IM in memory). Particulary useful for F\+L\+IM.
\item Platform independent C/\+C++ library with interfaces for scripting libraries
\end{DoxyItemize}

\subsection*{Capabilities}


\begin{DoxyItemize}
\item Fast (IO limited) Reading T\+T\+TR files
\item Generation / analysis of fluorescence decays
\item Time window analysis
\item Correlation of time event traces
\item Filtering of time event traces to generate instrument response functions for fluorescence decays analysis without the need of independent measurements..
\item Fast photon distribution analysis
\item Fast selection of photons from a photon stream
\end{DoxyItemize}

Generation of fluorescence decay histograms chinet outperforms pure numpy and Python based libraries by a factor of $\sim$40

\subsection*{Implementation}

Pure pure C/\+C++ and C\+U\+DA based high performance algorithms for real-\/time and interactive analysis of T\+T\+TR data.

\section*{Building and Installation}

\subsection*{C++ shared library}

The C++ shared library can be installed from source with \href{https://cmake.org/}{\tt cmake}\+:


\begin{DoxyCode}
git clone --recursive https://github.com/tpeulen/chinet.git
mkdir chinet/build; cd chinet/build
cmake ..
sudo make install
\end{DoxyCode}


On Linux you can build and install a package instead (prefered)\+:

\subsection*{Python bindings}

The Python bindings can be either be installed by downloading and compiling the source code or by using a precompiled distribution for Python anaconda environment.

The following commands can be used to download and compile the source code\+:


\begin{DoxyCode}
git clone --recursive https://github.com/tpeulen/chinet.git
cd chinet
sudo python setup.py install
\end{DoxyCode}


In an \href{https://www.anaconda.com/}{\tt anaconda} environment the library can be installed by the following command\+: 
\begin{DoxyCode}
conda install -c tpeulen chinet
\end{DoxyCode}


For most users the later approach is recommended. Currently, pre-\/compiled packages for the anaconda distribution system are available for\+:


\begin{DoxyItemize}
\item Windows\+: Python 2.\+7, Python 3.\+7 (x64)
\item Linux\+: Python 2.\+7, Python 3.\+7 (x64)
\item Mac\+Os\+: Python 2.\+7 (x64)
\end{DoxyItemize}

Legacy 32-\/bit platforms are not supported.

\subsection*{Examples}


\begin{DoxyCode}
\end{DoxyCode}


\subsection*{License}

chinet is released under the open source M\+IT license. 